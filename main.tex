\documentclass{homework}
\author{Maya Basu}
\class{Differential Topology}
\title{Class Notes}

\newcommand{\inte}{\text{Int}}
\newcommand{\clos}[1]{\overline{#1}}
\newcommand{\RR}{\mathbb{R}}
\newcommand{\BB}{\mathbb{B}}
\newcommand{\OO}{\mathbb{O}}

\begin{document} \maketitle

\section{Definition of a Topological Space, Open Sets, Closed Sets}

A topological space $(X, \OO)$ is a set $X$ and a collection $\OO$ of subsets of $X$ where $O \in \OO$ is called an "open set" such that
\begin{itemize}
\item{The union of any number of elements of $\OO$ is also an element of $\OO$}
\item{The intersection of any finite number of elements of $\OO$ is also an element of $\OO$}
\item{$\emptyset , X \in \OO$}
\end{itemize}
The condition that only a finite number of intersections can be allowed is illistrated by $\cap_{n = \infty} \left( -\frac{1}{n},\frac{1}{n}\right) = \{0\}$

A closed set is a set $S \subset X$ such that $X-S$ is open.

\subsection{Examples of Topological Spaces}
\begin{itemize}
\item{$\mathbb{R}$ with standard topology $\rightarrow$ standard notion of open subsets are open}
\item{$X$ is any set, $\OO = \{\emptyset,X\}$. This is called the "trivial topology"}
\item{$X$ is any set and $\OO$ is the set of all subsets of $X$. This is called the "discrete topology"}
\item{$X = \{1,2\}$ and $\OO = \{\emptyset, \{1\}, \{1,2\}\}$ is a valid topological space}
\end{itemize}

\section{Definition of continuity}
A function $f: X \rightarrow Y$ is continuous if for each open set $O$ in $Y$, $f^{-1}(O) = \{x \in X | f(x) \in O\}$ is also open in $X$.
\subsection{Examples involving the Continuity of Maps}
\begin{itemize}
\item{Suppose $(X,\OO_x)$ is a space with discrete topology and $(Y,\OO_y)$ is any topological space. Then any map $f: X \rightarrow Y$ is continuous}
\item{Suppose $(X,\OO_x)$ is a trivial topology. Then a map $f: X \rightarrow \mathbb{R}$ is only continuous if it maps each $x \in X$ to a single point in $R$}
\item{Let $X = \{x_1,x_2\}$ with discrete topology. $f: \mathbb{R} \rightarrow X$ is continuous iff $f$ maps $\mathbb{R}$ to one point in $X$. (The only sets both open and closed in $\mathbb{R}$ are $\emptyset$ and $\mathbb{R}$). Question: why is this equivalent to the intermediate value theorem?}
\end{itemize}

\section{Definition of a Neighborhood of x}
A Neighborhood of an element $x \in X$ is a subset $N \subseteq X$ such that there exists an open $O \subseteq X$ where $x \in O \subseteq N$. 

\section{Interiors and Closures}
Let $S$ be a subset of the topological space $(X,\OO_x)$. 
\subsection{Definition of an Interior}
$\inte(S) = \cup_{O \subseteq S | O \in \OO_x}O$
$\inte(S)$ (which is open as it is the union of open subsets) is the largest open subset in $S$ since if there is a hypothetical larger open subset in $S$ we know that it is actually contained in the union which constructs $\inte(S)$.

\subsection{Proof: $S$ is open iff $\inte(S) = S$}
$\inte(S) \subseteq S$. Additionally, if $S$ is open then since $S \subseteq S$, $S  \subseteq \inte(S)$. So if $S$ is open then $\inte(S) =S$. Going the other way, if $S = \inte(S) $ then $S$ is open as the union of open subsets of $S$.

\subsection{Definition of a Closure}
$\clos{S} = \cap_{S \subseteq C | C \text{ is closed in X}}$
\subsection{Proof that $\clos{S} = X - \inte(X - S)$}

By definition, $X - \clos{S} = X -  \cap_{S \subseteq C | C \text{ is closed in X}} C = \cup_{S \subseteq C | C \text{ is closed in X}}(X - C) = \cup_{O| O = X - C \text{is open} }O$. Each $X - C$ is an open subsets of $X - S$ ($S \subset C$ so $X - C \subseteq X - S$). Furthermore, for every open subset $O'$ of $X -S$, $X - O'$ is a closed set with $S \subseteq X - O'$ since any point in $S$ is not in $X - S$ and $O' \subseteq X - S$ so any point in $S$ is not in $O'$. $S \subseteq X$, so $S \subset X - O'$. Thus$X - \clos{S} = \cup_{O| O = X - C \text{is open} }O = \inte{X - S}$

\question Why is $\clos{S}$ the smallest closed set containing $S$?

\subsection{Definition of a Boundary}

$\partial S = \clos{S} - \inte(S)$ is the boundary of $S$. Since $\partial S = \clos{S} - \inte(S) = \clos{S} \cap (X - \inte(S)$ is an intersection of two closed sets it is closed

\section{Basis of Topology}
A basis of topology is a collection $\BB$ of subsets of $X$ satisfying:
\begin{itemize}
    \item For every point $x \in X$ there exists a $B \in \BB$ such that $x \in B$. 
    \item For every pair $B_1, B_2 \in \BB$ and each point $x \in B_1 \cap B_2$ there exists a $B_3 \in \BB$ such that $x \in B_3 \subseteq B_1 \cap B_2$.
\end{itemize}

\subsection{Proof: If $\BB$ is a basis of topology, then $(X, \OO)$ where $\OO$ is the set of all unions of sets in $\BB$ is a topological space. }

For any $O_1,O_2 \in \OO$, $O_1 \cup O_2$ is also in $\OO$ (still a union of sets in $\BB$). Additionally, consider a $x \in O_1 \cup O_2$. Then $x \in B_1$ and $x \in B_2$ for some $B_1 \subseteq O_1$ and $B_2 \subseteq O_2$. Then $x \in B_1 \cap B_2$ so there exists a $B_3 \in \BB$ such that $x \in B_3$ and $B_3 \subseteq B_1 \cap B_2$. So taking the union of all of these subsets $B_3$ (one for each $x \in O_1 \cap O_2$) we find that $O_1 \cap O_2$ is open as it is the union of sets in $\BB$. 

By induction, a finite number of intersections of sets in $\OO$ is in $\OO$, so then $(X, \OO)$ is a topological space. 

\subsection{Proof: $O \in \OO_{\BB}$ (the set of open sets as defined by a topological basis $\BB$) iff for every $x \in O$ there exists a $B_x \in \BB$ such that $x \in B_x \subseteq O$}

If the given statement is true then $O$ can be constructed as the union of the sets $B_x$ so it is open by definition. Conversely, if $O$ is open then it is equal to a union of subsets $B_i \in \BB$. Each point in $O$ then is inside one of these $B_i$ which is open by definition.

\section{Definition of a Metric Space}

A metric space is a pair $(X,d)$ with $d: X \times X \rightarrow \RR$ where:
\begin{itemize}
    \item $d(x,y) = d(y,x)$
    \item $d(x,y) \ge 0$, equality holds iff $x = y$
    \item $d(x,y) \leq d(x,z) + d(z,y)$
\end{itemize}

$B_r(x) = \{y \in X|d(x,y) <r\}$ is an open ball of radius $r$ in the metric space $(X,d)$.

For any metric space the collection of all open balls in a basis of topology.

A subset $S$ of a metric space is open iff for every $x \in S$, then there exists $r>0$ such that $B_r(x) \subseteq S$

\section{Subspaces}


Consider a topological space $(X, \OO)$ with a subset $Y \sebseteq X$. Then $O \subseteq Y$ is open in the subspace topology if there exists a $O' \subseteq X$ with $O = A \cup O'$.

We can obtain a basis of topology for $Y$ by taking $B_{Y_i} = B_i \cup A.$

If $X$ is a metric space we can obtain a metric $d_Y$ on $Y$ by restricting the domain of $d$ to $Y$  so that $(Y,d_Y)$ is also a metric space.

The two ways of assigning a topology on the subspace of a metric space (Metric space defines topology $\rightarrow$ subset topology and Metric space gives a metric on the subset $\rightarrow$ topology by metric induced on subset) give the same topology.




\subsection{Alternate View of the Subspace Topology}

Consider the inclusion map $i: Y \rightarrow X$ which sends $y_{\text{in y}} \rightarrow y_{\text{in X}}$. We define $\OO_y$ as the smallest (as a set) topology such that $i$ is continuous. 

Question: why the smallet? Discrete topology would not be unique - why?

\subsubsection{Proving existance of such a topology:}
For every $O \subset X$ where $O$ is open we have that $i^{-1}(S)$


\subsection{Disjoint Union}
Two seperate unattached subspaces sometimes the coproduct.
Take $X_1,X_2$ to be topological spaces with no shared elements. Then we consider the disjoint union $X_1 \sqcup X_2$ (union in the sense of set theory). We wish to make this disjoint union a topology. Consider the inclusion maps
\[i_k : X_k \rightarrow X_1 \sqcup X_2, k \in \{1,2\}\]
We want a topology on $X_1 \sqcup X_2$ such that $i_k$ is continuous and the topology is the largest possible (Question: why largest?).

If $i_k$ is continuous and $O$ is open in $X_1 \sqcup X_2$ then the preimages $O \cap X_1$ and $O \cap X_2$ must be open in $X_1$ and $X_2$ respectively. So all subsets that could be open in $X_1 \sqcap X_2$ are
\[\{O \subseteq X_1 \sqcup X_2 |O \cap X_1 \text{ and } O \cap X_2 \text{ are open }\}\]

Example: If $X$ is a finite (as a set) discrete topological space then $X = \sqcup_{x \in X} \{x\}$ where $\{x\}$ have the unique topology (only one element).



\section{Products}

Let $X_1,X_2$ be topological spaces, we consider $X_1 \times X_2$. 

\subsection{Projections}
The projection maps $P_k:X_1 \times X_2 \rightarrow X_k$ for $k \in \{1,2\}$, $P_k(x_1,x_2) = x_k$. 

\subsection{Topology on a product space}
For the topology on $X_1 \times X_2$ we pick the smallest topology such that $P_1$ and $P_2$ are continuous

\question Why do we pick the smallest topology?

If $P_1$ and $P_2$ are continuous then for an open set $O_{X_1}$ in $X_1$ and an open $O_{X_2}$ in $X_2$ we must have that $P_1^{-1} = O_{X_1} \times X_2$ is open and that $P_2^{-1} = X_1 \times O_{X_2}$ is open. Taking the intersection of $O_{X_1} \times X_2$ and $X_1 \times O_{X_2}$ we get $O_{X_1} \times O_{X_2}$ is open. 


The collection $\{O_{X_1} \times O_{X_2} | O_{X_k} \text{ is open in } X_k$ is not a topology, however it is a basis of topology - the "product topology".

For example, $\RR^2$ has a topologies defined in $2$ ways, using the euclidian metric, and the product topology of the standard topologies of $\RR$.

\subsubsection{The euclidean metric $\OO_{st}$ and the product topologys $\OO_P$ on $\RR$ are the same}

The basis given by the standard topology $\BB_{st}$ is the collection of open balls.

The basis given by the product topology $\BB_P$ is the collection of products of open subsets in $\RR$.

Now consider an element of $\BB_{st}$ which is an open ball, and a point $B \in B_{st}$. We can surround $B$ by an open rectangle entierly in $B_{st}$ so open balls are open in the product topology. Similarly, given an element $B_{pr} = O_1 \times O_2 \times \cdots O_n$ with all open $O_i$, we take a point in this subest and surround it by an open ball entirely within $ O_1 \times O_2 \times \cdots O_n$. For each $i$ there exists $r_i > 0$ such that $B_{r_i}(y_i) \subseteq O_i$ so then $B_{r_1}(y_1) \times B_{r_2}(y_2) \times \cdots B_{r_n}(y_n) \subseteq O_1 \times O_2 \times \cdots O_n$. Taking $R = \min r_i$ we get $B_R(\overarrow{y}) $ is a subset of the set open int he product topology. Thus, the basis in the product toplogy are open in the standard topolgy so, combined with the previous argument, these are the same topologies. 










\section{Definition of a Homeomorphism}

Definition 1: A Homeomorphism $f:X \rightarrow Y$ such that 
\begin{itemize}
    \item {f is a bijection}
    \item{For any $S \subseteq X$, $f(S)$ is open iff $S \in \OO$. (Equivalently, $f^{-1}(S)$ is open iff $S$ is open}
\end{itemize}

Definition 2: $f: X \rightarrow Y$ is a homeomorphism if $f$ is invertable and both $f$, $f^{-1}$ are continuous. 

We must require that $f^{-1}$ is also continuous? Yes - here is a non-example

Consider $f(x) = x$ for $x \in [0,1)$ and $f(x) = 1$ for $x \in \{2\}$ as it maps $[0,1) \cup \{2\}$ to $[0,1]$. This map is continuous and a bijection, however it's inverse is not continuous.

However, a function like $f(x) = tan\left\frac{\pi}{2}x\right)$ is a homemorphism from $(-1,1)$ to $\RR$ since the inverse function, $f^{-1} = \arctan(y) \cdot \frac{2}{\pi}$ is also continuous




\subsection{Properties of Homeomorphisms}

\begin{itemize}
    \item $X$ is homeomorphic to $X$ - we use the identity function and the same topology on both copies of $X$. 
    \item $X$ is homeomorphic to $Y$ iff $Y$ is homeomorphic to $X$ - if $f$ is a homeomorphism then $f^{-1}$ exists and is also a homeomorphism.
    \item If $X$ is homeomorphic to $Y$ and $Y$ is homeomorphic to $Z$ then $X$ is homeomorphic to $Z$ 
    
    Proof: 
    
    Since $f: X \rightarrow Y$ and $g: Y \rightarrow Z$ are continuous maps their composition is continuous since for any open subset $O_z$ of $Z$, $g^{-1}(O_z)$ is an open subset of $Y$, so $f^{-1}g^{-1}(O_z)$ is an open subset of $X$. Additionally, $f^{-1}$ and $g^{-1}$ exist and are continuous so $f \circ g$ is continuous and $(f \circ g)^{-1}$ is continuous. Finally, the existance of $(f \circ g)^{-1}$ implie $f \circ g$ is invertable so $X$ is homeomorphic to $Z$.

    

    
\end{itemize}


\section{Connectedness}

If $X$ satisfies any one of the following (equivelent) properties then it is disconnected. If $X$ is not disconnected, it is connected. Being connected or not depends only on topological type, it does not change under homeomorphism.

\begin{enumerate}
    \item There is a continuous, non-constant map $f:X \rightarrow \{1,2\}$ where $\{1,2\}$ has discrete topology.
    \item $X$ is homeomorphic to $Y \sqcup Z$, $Y$ and $Z$ are nonempty topological spaces
    \item {There are nonempty open sets $O_1,O_2 \subset X$ such that $O_1 \cup O_2 = X$ and $O_1 \cap O_2 = \emptyset$}
    \item {There are nonempty closed subsets $C_1,C_2 \subset X$ such that  $C_1 \cup C_2 = X$ and $C_1 \cap C_2 = \emptyset$}
    \item{There is a subset $S \subset X$, $S \neq \emptyset$ and $S \neq X$ that is both closed and open}
    
\end{enumerate}

$3$, $4$, and $5$ are clearly equivalent using the fact that the complement of an open set is closed. 

$3$ implies $2$ by giving $O_1,O_2$ subset topology and choosing the identity map as the homeomorphism and $2$ implies $3$ by taking the disjoint sets in $3$ to be the preimages of the two spaces from $2$.

$3$ implies $1$ since we can take the (continuous) map $O_i$ to $i$, and to get from $1$ to $3$ we define $O_i$ as the preimage of $i$. 

\subsection{Path Connectedness}

Definition: $X$ is path connected if for any points $a,b$ in $X$ there is a path connecting $a$ and $b$ in $X$. 


\subsection{A path connected space is connected}

Suppose that $X$ is a path connected but disconnected space. Then let $O_1,O_2 \subset X$ be non empty sopen subsets satisfying $3$ from above. Since these sets are non empty there exist $a \in O_1$ and $b \in O_2$ with a path connecting them. Let us take $\gamma: [0,1]$ in $X$ connecting $a$ and $b$. Then $S_1 = \gamma^{-1}(O_1)$ is open since $\gamma$ is continuous, and nonempty since $\gamma(0) = a \in O_1$. Similarly, $S_2 = \gamma^{-1}(O_2)$ is also open and nonempty. Additionally, $S_1 \cup S_2 = [0,1]$ and $S_1 \cap S_2 = \emptyset$ so $[0,1]$ must be disconnected. But $[0,1]$ is connected leading to a contradiction. 


\subsection{$\mathbb{R}^n$ is connected}
$\RR^n$ is path connected since for $\overarrow{a},\overarrow{b}$ we can take $\gamma(t) = (1 - t)\overarrow{a} + t \overarrow{b}$. Similarly, any convex subset of $R^n$ is connected, specifically open balls. 


Are all connected spaces path conncted? No, one example is $\{(x,\sin\left(\frac{1}{x}\right) | x \in (0,\infty)\} \cup \{(0,y) | y \in [1,-1]\}$.

Question: why is this an example?



\subsection{$[0,1]$ is connected}

We will prove this by showing equivalence to the intermediate value theorem (let $f:[a,b] \rightarrow \RR$ be a continuous map sith $f(a) < 0 < f(b)$ then $f(c) = 0$ for some $c$). 


Assume that the intermediate value theorem is false and theri exists a counter example $f$. Then let $O_1 = f^{-1}(-\infty,0)$ and $O_2 = f^{-2}(0,\infy)$. Since $f$ is continuous both of these sets are open and $O_1 \cap O_2 = \emptyset$ and $O_1 \cup O_2 = [a,b]$ since $0$ is not a value of $f$. This implies that $[a,b]$ is disconnected by $3$.

We prove the other direction. if $[0,1]$ is disconnected consider the continuous, non constat map $\sigma: [0,1]$. Since $\sigma$ 




\section{Path Connected}
Points $a$ and $b$ in a topological space can be connected with a map $\gamma : [0,1]$ where $\gamma(0) = a$ and $\gamma(1) = b$. 

Being path connected is an equivelance relation:
\begin{itemize}
    \item $a$ is path connected to $a$ (Take the constant map)
    \item If $a$ is connected to $b$ then $b$ is connected to $a$ (Given $\gamma : [0,1] \rightarrow X$, $\gamma (0) = a$, $\gamma (1) = b$, take $\gamma (1-t)$ or mor formally take $I : [0,1] \rightarrow [0,1]$, $I(t) = 1- t$. $I$ is continuous so $\gamma \circ I$ is continuous)
    \item If $a$ is connected to $b$ and $b$ is connected to $c$ then $a$ is connected to $c$

\end{itemize}

Question: How to prove this last relation?


Any topological space splits (as a set) into path connected components so we can write $X = \cup C_i$ where $C_i$ is connected and $C_i \cap C_j = \emptyset$ if $i \neq j$. However, $X \neq \sqcup C_i$ as a topological space. (Question: why not?)

Question: what does the axiom of choice have to do with finiteness



http://individual.utoronto.ca/aaronchow/notes/mat327h1.pdf 

\end{document}